\documentclass[12pt,letterpaper]{article}

% =====================
% Configuración general
% =====================
\input{config/libs.tex}
\input{config/format.tex}

% =====================
% Documento
% =====================
\begin{document}
\renewcommand{\figurename}{Ilustración} 
{\textbf{\textcolor{azulOscuro}{INFORME TAREA 2.1}}}

%===========================================================
%===========================================================
% --- Tabla de Datos ---
%===========================================================
%===========================================================

{\setlength{\parskip}{0pt}
\section{Datos Generales}
}

\begin{tabularx}{\textwidth}{|>{\raggedright\arraybackslash}X|>{\raggedright\arraybackslash}X|}
\hline
Título del Informe: & Consumo de APIS+Arquitectura Limpia + Provider \\
\hline
Autor(a): & Carlos Hernández, Olivier Paspuel, Antonio Revilla, Frederick Tipán\\
\hline
Carrera: & Ingeniería en Software \\
\hline
Asignatura o Proyecto: & Desarrollo de Aplicaciones Móviles \\
\hline
Tutor o Supervisor: & Mgtr. Doris Karina Chicaiza Angamarca\\
\hline
Institución: & Universidad de las Fuerzas Armadas ESPE – Matriz Sangolquí \\
\hline
Fecha de entrega: & 30 de noviembre de 2025 \\
\hline
\end{tabularx}


%===========================================================
%===========================================================
% --- Introducción ---
%===========================================================
%===========================================================

\section{Introducción}

El siguiente informe presenta el desarrollo de una aplicación móvil, la cual va a consumir una API externa de películas para poder ver el contenido de forma dinámica. Para el desarrollo de esta aplicación móvil se trabajó aplicando una arquitectura limpia, para poder garantizar una separación de responsabilidades entre las diferentes capas que se van a utilizar.

Se implementó el patrón Provider para la gestión de estados de la aplicación. Este patrón permite administrar de forma eficiente y reactiva los datos. Adicional, también se aplicó el patrón de diseño Atomic Design para el diseño de la interfaz, para generar una modularización y asegurar una consistencia visual, pero sobre todo la reutilización de código para evitar duplicidad. Además, se integraron librerías para las peticiones de red y para la parte visual; de forma específica se utilizó la librería http y font\_awesome\_flutter.

Esta aplicación permitió familiarizarse de mejor manera con las técnicas necesarias para integrar servicios RESTful en aplicaciones móviles desarrolladas con Flutter. Esta aplicación permite comprender de mejor manera el proceso de construcción de una aplicación escalable y mantenible, siguiendo estándares que se utilizan en el desarrollo moderno.

\newpage

%===========================================================
%===========================================================
% --- índice de Contenidos y Figuras ---
%===========================================================
%===========================================================

\renewcommand{\contentsname}{Índice de Contenidos}
{\setlength{\parskip}{0pt}
\tableofcontents
}

\vspace{1.0cm}

\renewcommand{\listfigurename}{Índice de Ilustraciones}
{\setlength{\cftbeforefigskip}{2pt}
\listoffigures
}

\newpage

%===========================================================
%===========================================================
% --- Objetivos ---
%===========================================================
%===========================================================

\section{Objetivos}
\subsection{Objetivo General}
Desarrollar una aplicación que consuma una API externa de películas, la cual implemente una arquitectura limpia y el patrón Provider, de manera que la aplicación resulte escalable, mantenible y facil de trabajar para el desarrollador.

\subsection{Objetivo Específicos}
- Estructurar las capas del proyecto aplicando Arquitectura Limpia, para garantizar la separación de responsabilidades y la mantenibilidad del código.

- Gestionar el estado reactivo de la aplicación implementando el patrón Provider para un flujo de datos eficiente entre los componentes visuales.

- Visualizar información dinámica de películas consumiendo una API RESTful mediante el uso de la librería http y modelos de datos personalizados.

%===========================================================
%===========================================================
% --- Marco Teórico ---
%===========================================================
%===========================================================

\section{Marco Teórico}
\subsection{Arquitectura Limpia (Clean Architecture)}
En un patrón de diseño de software que lo que intenta es estandarizar la forma en que se organizan los proyectos complejos, esta arquitectura busca que los frameworks, las interfaces de usuario y las bases de datos funcionen de manera independiente. Esta arquitectura se basa en un principio clave el cual se llama la “Regla de Dependencia”, la cual indica que las dependencias del código fuente solo deben orientarse hacia el interior. Esto implica que las capas internas no necesitan saber nada sobre las externas (\cite{martin2018clean}).

En el contexto de Flutter, esta arquitectura se suele dividir en tres capas fundamentales:
\begin{itemize}
    \item \textbf{Presentación (Presentation Layer):} contiene la interfaz de usuario (widgets) y los gestores de estado. Su función principal consiste en mostrar los datos al usuario y recoger sus interacciones, actuando como el punto de contacto directo con la aplicación.
    \item \textbf{Dominio (Domain Layer):} representa el núcleo de la aplicación. Incluye las entidades (modelos de negocio) y los casos de uso, y se mantiene totalmente independiente de librerías externas o detalles concretos de implementación, lo que le da mayor estabilidad a largo plazo.
    \item \textbf{Datos (Data Layer):} se encarga de obtener la información, ya sea desde una base de datos local o a través de una API remota. En esta capa se implementan los repositorios que fueron definidos previamente en la capa de dominio, cerrando así el ciclo entre la lógica de negocio y las fuentes de datos.
\end{itemize}

\subsection{Patrón de Gestión de Estado Provider}
Provider representa, una mejora sólida frente a los InheritedWidgets que Flutter ofrece de forma nativa. Su objetivo principal es simplificar de manera notable la inyección de dependencias y la gestión del estado, funcionando como un puente claro entre la lógica interna y la interfaz de usuario. Lo que realmente lo hace resaltar es su capacidad para separar los datos de la vista. Gracias a esto, la información puede circular con fluidez por el árbol de widgets y los cambios se notifican de forma inteligente, actualizando solo los componentes que realmente necesitan renovarse. Al final, esto se traduce en un mejor rendimiento de la aplicación (\cite{flutter2024provider}).

Para entender mejor cómo funciona, se deben analizar tres componentes esenciales.

\begin{itemize}
    \item \textbf{ChangeNotifier:} actúa como el núcleo central, donde se encuentran los datos y la lógica de negocio. Esta clase se encarga de almacenar el estado mutable, como podría ser una lista de películas, y de avisar a sus escuchas cada vez que ocurre un cambio.
    
    \item \textbf{ChangeNotifierProvider:} funciona como el canal de distribución. Este widget crea una instancia del ChangeNotifier y la inserta en el árbol de widgets, de modo que cualquier descendiente que la necesite pueda acceder a ella sin mayores complicaciones.
    
    \item \textbf{Consumer:} se comporta como el receptor atento. Se suscribe a las actualizaciones del proveedor y, cuando recibe una notificación, reconstruye solo la parte específica de la interfaz que depende de esos datos, evitando redibujar toda la pantalla de manera innecesaria.
\end{itemize}

\subsection{Consumo de Servicios RESTful y Protocolo HTTP}
La arquitectura REST, o Transferencia de Estado Representacional, plantea una forma ordenada de organizar la comunicación dentro de sistemas que estan distribuidos, centrándose principalmente en la manipulación de recursos que se identifican por medio de URIs y respaldada por un modelo cliente-servidor sin estado. De forma general esta arquitectura plantea interacciones que ocurren a través del protocolo HTTP, con solicitudes viajando de forma independiente y llevando toda la información requerida para que el servidor la procese. Esta arquitectura impulsa la escalabilidad y la solidez del sistema (\cite{richardson2007restful}).

Las operaciones sobre los recursos se expresan mediante métodos estandarizados en el RFC 7231 (2014), que marcan el tipo de acción que se desea realizar:
\begin{itemize}
    \item \textbf{GET:} recupera la representación de un recurso sin producir efectos secundarios, es decir, se limita a la lectura.
    \item \textbf{POST:} envía datos para procesar una entidad o crear un recurso nuevo cuando sea necesario.
    \item \textbf{PUT:} reemplaza por completo la representación actual de un recurso existente.
    \item \textbf{PATCH:} permite aplicar cambios parciales sobre un recurso ya creado.
    \item \textbf{DELETE:} solicita la eliminación del recurso especificado.
\end{itemize}

Consumir estos servicios implica trabajar de forma asíncrona. Como las operaciones de red introducen latencia, el cliente debe encargarse de procesar las respuestas y los códigos de estado sin bloquear el hilo principal de ejecución. De este modo, se mantiene el rendimiento y la fluidez de la aplicación, algo fundamental en sistemas interactivos y en tiempo real.

\subsection{Serialización y Mapeo de Datos (JSON)}
La serialización es el proceso mediante el cual se transforman estructuras de datos u objetos en un formato que pueda transmitirse o almacenarse, de forma que luego sea posible reconstruirlos. Este mecanismo resulta fundamental en sistemas distribuidos, porque permite el intercambio estructurado de información entre distintas aplicaciones (\cite{gamma1995design}).

En el desarrollo de software actual, el formato más utilizado para este intercambio es JSON (JavaScript Object Notation). Cuando una aplicación recibe datos en JSON, es necesario convertir ese texto en objetos internos que puedan manipularse con comodidad. En lenguajes fuertemente tipados, como Dart, C\# o Java, este proceso se conoce como deserialización o mapeo de datos y cumple un papel clave: garantiza la seguridad de tipos y ayuda a evitar inconsistencias en tiempo de ejecución. En entornos como Flutter, esta conversión suele implementarse a través de clases modelo que incluyen métodos o constructores especializados para interpretar el contenido del JSON.

\begin{center}
\begin{lstlisting}
class Movie {
  final int id;
  final String title;
  final String? posterPath;
  final double voteAverage;
  final DateTime releaseDate;

  Movie({
    required this.id,
    required this.title,
    this.posterPath,
    required this.voteAverage,
    required this.releaseDate
  });

  // Método de fábrica para crear una instancia desde un mapa JSON
  factory Movie.fromJson(Map<String, dynamic> json) {
    return Movie(
      id: json['id'],
      title: json['title'],
      posterPath: json['poster_path'],
      voteAverage: json['vote_average'].toDouble(),
      releaseDate: DateTime.parse(json['release_date']),
    );
  }
}
\end{lstlisting}
\end{center}

Este proceso de mapeo explícito ayuda a mantener una coherencia clara entre la estructura de datos que se recibe y los modelos internos de la aplicación, lo que a su vez favorece la mantenibilidad del proyecto, mejora la claridad del código y contribuye a que el sistema pueda escalar de manera ordenada.

%===========================================================
%===========================================================
% --- Desarollo ---
%===========================================================
%===========================================================

\section{Desarrollo}

El proyecto se encuentra cargado en el siguiente repositorio de GitHub: 

\href{https://github.com/Saint-Roche-Microsystems/Filmsit}{\color{blue}\underline{https://github.com/Saint-Roche-Microsystems/Filmsit}}

\subsection{Estrucutra de proyecto}

Se consume la API pública de The Movie Database (TMDb) para obtener información sobre películas, series de televisión y actores. La API proporciona datos detallados, incluyendo títulos, descripciones, calificaciones, imágenes y más.

La app utiliza el patrón arquitectónico MVVM en conjunto con Clean Architecture para organizar su distribución.Para ello, se parte de la carpeta “src/”, la cual se compone de los siguientes directorios:

\begin{figure}[H]
    \centering
    \includegraphics[width=0.7 \textwidth, height=10cm, keepaspectratio]{estructura_proyecto.png}
    \caption{Estructura de directorios}
    \label{fig:estructura_proyecto}
\end{figure}

Siendo los directorios princiales los siguientes:

\begin{itemize}
    \item \textbf{core:} Contiene la configuración principal de la aplicación, donde se optienen las variables de entorno. También se definen manejadores de errores simples, utilizados en llamadas de API u otros errores en la aplicación. 
    
    \item \textbf{data:} Implementa la capa de datos, incluyendo modelos, fuentes de datos (en este caso remotas por la API) y repositorios que gestionan la obtención y almacenamiento de datos.
    
    \item \textbf{domain:} Define la lógica de negocio a través de entidades, casos de uso a partir de los repositorios abstractos que describen las operaciones disponibles.
    
    \item \textbf{presentation:} Maneja la interfaz de usuario, el ruteo y la interacción del usuario mediante pantallas, widgets y proveedores para el estado reactivo con view models.
    
    \item \textbf{themes:} Contiene los esquemas de colores, tipografía y estilos reutilizables para mantener la consistencia visual en toda la aplicación.
\end{itemize}

\subsection{Entidades}

Se implementaron tres tipos principales de entidades, por consiguiente se cuenta tanto con una declaración tanto en el datasource, repository y model.

\begin{itemize}
    \item \textbf{Genre:} Estructura simple en la que se especifica el tipo de géneros disponibles en la API. Se compone de un \lstinline{id} id y \lstinline{name}.
    \item \textbf{Movie:} Entidad principal en la que seobtiene información básica de cada película para presentarlo en forma de resumen en la UI. Se compone de \lstinline{id}, \lstinline{title}, \lstinline{posterPath}, \lstinline{voteAverage} y \lstinline{releaseDate}.
    \item \textbf{MovieDetails:} Contiene atributos restantes que complementan la información de cada película. En esta se agregan los atributos de \lstinline{overview}, \lstinline{genres} (lista de \lstinline{Genre}), \lstinline{backdropPath}, \lstinline{budget} y \lstinline{productionCompaniesNames} (lista de \lstinline{String}).
\end{itemize}

Cada entidad parte de endpoints diferentes para obtener los reultados, dichas declaraciones se encuentran explícitamente dentro de los \textbf{datasources}, en donde dependiendo del caso se necesita de tomar ciertos argumentos para poder hacer correctamente la llamada a la API. Así se tiene la siguiente declaración de métodos en la interfaz abstracta:

\begin{center}
\begin{lstlisting}
abstract class BaseMovieDataSource {
  Future<List<MovieModel>> getTrendingMovies();
  Future<List<MovieModel>> getUpcomingMovies();
  Future<List<MovieModel>> getPopularMovies({int page = 1});
  Future<List<MovieModel>> getMoviesByGenre(int genre, {int page = 1});
  Future<List<MovieModel>> searchMovies({required String querry});
  Future<MovieDetailsModel> getMovieDetails({required int id});
}

abstract class BaseGenreDataSource {
  Future<List<GenreModel>> getGenres();
}
\end{lstlisting}
\end{center}

Como se da a notar, se tienen dos clases abstractas que permiten dividir la responsabilidad y las llamadas en base al tipo de entidad.
\subsection{Repositorios}

Cada repositorio implementa su interfaz correspondiente en base al tipo de entidad. Así mismo, cada respositorio hace uso de su respectivo \textbf{datasource} para obtener los datos.

En algunos casos simplemente se hace la llamada a los datos y se toma solo los primeros 3 o 5 elementos, pero en otros casos donde se toman más datos, se optó por utilizar las siguientes reglas de paginación:

\begin{itemize}
    \item \textbf{Parámetros de entrada:} Se recibe de parámetro la varaible \texttt{page}, la cual por defecto siempre será 1.
    
    \item \textbf{Cantidad de resultados:} Se obtienen resultados en bloques de 10, por lo que por cada página de la API se obtienen 2 bloques de resultados.
\end{itemize}

Puesto que la variable \lstinline{page} que viene como parámetro no es precisamente la misma de la API, se tiene que dividir los resultados en base a si la página es par o impar, siguiendo el procedimiento:

\begin{center}
\begin{lstlisting}
      int pageIndex = (page ~/ 2);
      if(page%2 != 0) {
        pageIndex++;
      }

      // Hacer la consulta a la API...

      // Odd numbers: 1-10
      if(page%2 == 0) {
        // Retornar primer grupo
      }

      // Even numbers: 11-20
      return // Retornar segundo grupo
\end{lstlisting}
\end{center}

\subsection{Casos de Uso}

Los casos de uso van de la mano con los métodos descritos tanto en los repositorios como en los datasources. A continuación se muestra los archivos de todos los casos de uso.

\begin{figure}[H]
    \centering
    \includegraphics[width=0.5 \textwidth, height=6cm, keepaspectratio]{casos_de_uso.png}
    \caption{Archivos de casos de uso}
    \label{fig:use_case_files}
\end{figure}

Cabe mencionar que internamente, cada caso de uso hace una llamada a un método en específico del repositorio. Posteriormente este caso de uso es referenciado en los ViewModels para diversas funciones de la aplicación.

\subsection{Declaración de Providers}

Estos se ubican dentro del archivo \lstinline{main.dart}. Se utiliza la librería “Provider” para la gestión del estado de la aplicación. Primero se parte de la definición de los providers http, datasource y repositories.

\begin{center}
\begin{lstlisting}
  final httpClient = http.Client();
  final movieDs = MovieDataSource(client: httpClient);
  final movieRep = MovieRepository(ds: movieDs);
  final genreDs = GenreDataSource(client: httpClient);
  final genreRep = GenreRepository(ds: genreDs);
\end{lstlisting}
\end{center}

Es en base a estos que luego se pueden definir los providers para cada uno de los casos de uso siguiendo la estructura:

\begin{center}
\begin{lstlisting}
Provider<CASO_DE_USO>(
    create: (_) => CASO_DE_USO(REPOSITORIO),
),
\end{lstlisting}
\end{center}

Una vez definidos los provideres de los casos de uso, se los suministra a los viewmodels correspondientes mediante los \lstinline{ChangeNotifierProvider} siguiendo la estructura:

\begin{center}
\begin{lstlisting}
ChangeNotifierProvider<VIEW_MODEL>(
  create: (context) => VIEW_MODEL(
    useCase1: context.read<useCase1>(),
    useCase2: context.read<useCase2>(),
    ...
  ),
),
\end{lstlisting}
\end{center}

\subsection{Ruteo}

La aplicación cuenta con dos rutas principales, las cuales se encuentran declaradas en el archivo \lstinline{app\_routes.dart}:

\begin{itemize}
    \item \textbf{home: } Utiliza la ruta “/”, siendo una ruta estática que llama a el widget \lstinline{HomeScreen}.
    
    \item \textbf{movieDetail: } Utiliza la ruta “/movie-detail” , siendo una ruta dinámica que llama al widget \lstinline{MovieDetailScreen}. También verifica que se le haya pasado un argumento de tipo \lstinline{int} que hace referencia al id de la película para poder obtener sus resultados y poder mostrarlos en pantalla.
\end{itemize}

Adicionalmente, se cuenta un caso donde se accede a una ruta inválida o no especificada, en estos caso se define una pantalla de error simple \lstinline{Errorpage} que muestra un mensaje indicando el error junto con un botón para volver a la pantalla anterior.

\subsection{View Models}

Existen 3 tipos de viewmodels, una para cada entidad. Estos generalmente se componen de:

\begin{itemize}
  \item \textbf{atributos: } Son los casos de uso que permiten ejecutar la consulta a la API y obtener los datos procesados.

  \item \textbf{estados: } Son variables adicionales que construyen los elementos de la UI. Generalmente se tiene una lista con los datos, un booleado para mostrar un estado de carga y un string para mostrar mensajes de error. En otros casos se agregan otras varaibles, generalmente para valores de elementos de formularios.

  \item \textbf{métodos: } Son funciones que permiten actualizar el estado de las varaibles, lo que sumado con los providers permite actualizar la UI de forma reactiva entre distintos componentes. Para esto es fundamental utiliza \lstinline{notifyListeners()}.
\end{itemize}

Luego, estos ViewModels son utilizados en distintos widgets y pantallas mediante el widget \lstinline{Consumer}, lo que permite acceder a las varaibles de estado y establecer el comportamiento de la UI. Así se tienen las siguientes pantallas principales:

\textbf{HomePage}

Esta pantalla se compone de una barra de búsqueda en la parte superior. Luego tiene tres secciones:

\begin{itemize}
  \item \textbf{Trending: } Muesta un carrusel de tarjetas de películas en tendencia. Muestra las 3 más populares.

  \item \textbf{Upcomming: } Muestra una lista Horizontal de 5 películas que están próximas a estrenarse.

  \item \textbf{Discover: } Se compone de un \lstinline{DropDown} que sirve como filtros de películas basados en su género. Luego de un \lstinline{GridView} que muestra las películas que cumplen con el filtro en una cuadrilla de 5x2. Finalmente se pueden ver nuevas películas con un elemento de \lstinline{Pagination} al final de la lista.
\end{itemize}


\textbf{MovieDetailsPage}

En esta se muestra datos específicos de una película. Se muestra el póster, título, calificación, fecha de estreno, géneros, sinopsis y empresas productoras. Todo esto mediante una organización separando cada sección en los siguientes widgets: \lstinline{MovieSliverAppBar}, \lstinline{DetailsMainSection}, \lstinline{MetadataSection}, \lstinline{GenresSection}, \lstinline{OverviewSection}, \lstinline{BudgetSection} y \lstinline{ProductionSection}.


\subsection{Widgets}

Estos widgets fueron separados en base a componentes mas grandes, tratando de hacer que sean lo más resutilziables posibles. En varios casos se juntan para poder generar partes de una sección o página más grande.

\begin{figure}[H]
    \centering
    \includegraphics[width=0.6 \textwidth, height=9cm, keepaspectratio]{widgets.png}
    \caption{Archivos de widgets}
    \label{fig:project_widgets}
\end{figure}

Entre los widgets más importantes se encuentran:

\textbf{Pagination}

Este widget se compone de botones más pequeños, los que corresponden a flechas de atras y adelante y los botones con los números de página.

Para este caso se limitó al widget para que solo pueda mostrar 3 botoes de página a la vez. Si se presionan otros botones o las flechas, se actualiza la página y se muestran los otros números. También se definió un máximo de 5 páginas para toda la aplicación.

\begin{figure}[H]
    \centering
    \includegraphics[width=0.5 \textwidth, height=6cm, keepaspectratio]{pagination_widget.png}
    \caption{Widget de paginación}
    \label{fig:pagination_widget}
\end{figure}


\textbf{SearchBar}

Esta barra de búsqueda se compone de dos elementos. Por un lado se tiene el campo de texto donde se ingresa un texto a buscar. Este campo es controlado por una función \lstinline{debounce} en el viewmodel, para evitar hacer muchas peticiones a la API mientras el usuario escribe.

Por otro lado, se tiene un \lstinline{Overlay} que despliega una lista con los resultados obtenidos al hacer la consulta. Esta se ubica en la parte inferior del searchbar.

\begin{figure}[H]
    \centering
    \includegraphics[width=0.5 \textwidth, height=7cm, keepaspectratio]{search_widget.png}
    \caption{Widget de barra de búsqueda}
    \label{fig:searchbar_widget}
\end{figure}

\textbf{FilterDropdown}

Este widget muestra los géneros por los cuales se puede filtrar la lsita de películas. Para ello, se tiene un campo \lstinline{value} de tipo entero que corresponde al id del género dado por la API. Si se tiene un valor de 0 se aplica sin filtros.

\begin{figure}[H]
    \centering
    \includegraphics[width=0.5 \textwidth, height=7cm, keepaspectratio]{filter_widget.png}
    \caption{Widget de lista de filtros}
    \label{fig:filter_dropdown_widget}
\end{figure}

\textbf{GlassContainer}

Es un contenedor genérico que aplica un efecto de vidrio esmerilado a su contenido. Este widget se utiliza en varias partes de la aplicación para mejorar la apariencia visual y resaltar ciertos elementos. En concreto su apariencia se obtiene especificando el atributo de decoración del widget de \lstinline{Container} de la siguiente forma:

\begin{center}
\begin{lstlisting}
  decoration: BoxDecoration(
    color: SaintColors.surface.withValues(alpha: 0.7),
    borderRadius: borderRadius ?? BorderRadius.circular(16),
    border: Border.all(
      color: SaintColors.primary.withValues(alpha: 0.2),
      width: 1,
    ),
    boxShadow: [
      BoxShadow(
        color: Colors.black.withValues(alpha: 0.37),
        blurRadius: 32,
        offset: const Offset(0, 8),
      ),
    ],
  ),
\end{lstlisting}
\end{center}

\textbf{MovieCard}

Este es el elemento mas notable de la aplciación. En escencia solo muestra una tarjeta con la imagen de la película y su título en la parte inferior. El tamaño y su contenido se adaptan al espacio disponible debido a que se utiliza el widget \lstinline{Expanded}.

Adicioanlmente cuenta con dos elementos visuales más que se pueden especificar si se los quiere mostrar o no:

\begin{itemize}
  \item \textbf{score circle: } Un círculo que muestra la puntuación de la película en la parte inferior derecha sobre la tarjeta. Este cpirculo actualiza su color dependiendo del rango de la puntuación.
   
  \item \textbf{ribbon badge: } Una cinta en la parte superior derecha deforma inclinada que muestra un texto con un color especificado. Sirve para dar un realce a la tarjeta.
\end{itemize}

Dichos elementos se muestran por encima de la tarjeta gracias a que se los compone dentro del widget \lstinline{Stack}.

Adiconalmente, estas tarjetas pueden ser presionadas con un atributo \lstinline{onTap} que recibe una función para ejecutar cuando se presiona la tarjeta. 

\begin{figure}[H]
    \centering
    \includegraphics[width=0.6 \textwidth, height=8cm, keepaspectratio]{card_widget.png}
    \caption{Widget de tarjeta de película}
    \label{fig:movie_card_widget}
\end{figure}


Comúnmente el widget padre es el que suministra la función, para el caso donde se quiera pasar a la pantalla de detalles de la película, la cual es implementada mediante el siguiente código:

\begin{center}
\begin{lstlisting}
onTap: (movie) {
  Navigator.pushNamed(context,
    AppRoutes.movieDetail,
    arguments: MovieDetailsArguments(movieId: movie.id),
  );
},
\end{lstlisting}
\end{center}

Notar que el atributo \lstinline{movie} es generalmente dado por el viewmodel. Para pasar a la pantalla de detalle se especifica la ruta \lstinline{AppRoutes.movieDetail} y se pasan los argumentos necesarios mediante la clase \lstinline{MovieDetailsArguments}, en este caso es simple porque solo es el id.




%===========================================================
%===========================================================
% --- Conclusiones y Recomendaciones ---
%===========================================================
%===========================================================

\section{Conclusiones y Recomendaciones}
\subsection{Conclusiones}
\begin{itemize}
    \item Se logro desarrollar la aplicación con una arquitectura limpia, en la cual se estableció una clara separación de responsabilidades entre las capas de dominio, datos y presentación. El uso de la arquitectura limpia dentro del proyecto facilito la comprensión del flujo de trabajo y permitió tener un proyecto mantenible en el cual las modificaciones de la lógica de negocio no afecta a la interfaz de usuario.
    
    \item El uso del patrón Provider permitió la construcción de una aplicación eficiente, con una gestión de estados centralizada de la aplicación, esto desacopla la lógica de los componentes visuales, de manera que se logro que la interfaz reaccione fluidamente a los cambios de datos como la carga de los mismos, de manera que tiene un flujo de datos dinámico y escalable. 

    \item Se desarrollo de forma correcta la integración con una API externa RESTful específicamente una que tiene un catalogo de películas, además se integro el uso de la librería http. Se concluyo con una aplicación capaz de transformar las respuestas JSON en entidades personalizadas y además verlas en pantalla, logrando el objetivo de mostrar información de forma dinámica y real en la interfaz de usuario.
\end{itemize}

\subsection{Recomendaciones}
\begin{itemize}
    \item Considerando el uso de una arquitectura limpia en nuestro proyecto, seria algo recomendable implementar pruebas unitarias en la aplicación, de forma que se pueda asegurar la estabilidad del sistema en caso de que este quiera escalar, aplicar los unit testing permitirá validar de forma automática la separación de responsabilidad a medida que el proyecto vaya escalando.

    \item Se recomienda la aplicación de mecanismos de persistencia local utilizando caché dentro de los repositorios o viewmodel, esto permitirá potencia la gestión reactiva que se logra con el Provider. Esto haría que la aplicación puede mostrar contenido incluso sin conexión a internet, mejorando la satisfacción del usuario aprovechando la infraestructura de estado ya construida.
    
    \item Tomando la conexión que se realiza con una API externo, seria recomendable optimizar el consumo de datos aplicando técnicas de paginación infinita en lugar de una paginación manual, además de mejorar el manejo de errores de red con mensajes amigables para asegurar e informar sobre el rendimiento optimo en dispositivos con conexiones inestables.
\end{itemize}

%===========================================================
%===========================================================
% --- Referencias Bibliográficas ---
%===========================================================
%===========================================================

\section{Referencias Bibliográficas}
\printbibliography[heading=none]

%===========================================================
%===========================================================
% --- Anexos ---
%===========================================================
%===========================================================

\section{Anexos}

\textbf{Pantalla Home}

\begin{figure}[H]
    \includegraphics[width=0.8 \textwidth, height=12cm, keepaspectratio]{anexos/home.png}
\end{figure}

\textbf{Pantalla Detalle}

\begin{figure}[H]
    \includegraphics[width=0.7 \textwidth, height=10cm, keepaspectratio]{anexos/details.png}
\end{figure}

\textbf{Error al buscar}

\begin{figure}[H]
    \includegraphics[width=0.6 \textwidth, height=8cm, keepaspectratio]{anexos/no_results.png}
\end{figure}

\textbf{Widget de recarga}

\begin{figure}[H]
    \includegraphics[width=0.6 \textwidth, height=8cm, keepaspectratio]{anexos/refresh.png}
\end{figure}


\end{document}

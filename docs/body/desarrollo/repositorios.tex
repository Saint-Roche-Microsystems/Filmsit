\subsection{Repositorios}

Cada repositorio implementa su interfaz correspondiente en base al tipo de entidad. Así mismo, cada respositorio hace uso de su respectivo \textbf{datasource} para obtener los datos.

En algunos casos simplemente se hace la llamada a los datos y se toma solo los primeros 3 o 5 elementos, pero en otros casos donde se toman más datos, se optó por utilizar las siguientes reglas de paginación:

\begin{itemize}
    \item \textbf{Parámetros de entrada:} Se recibe de parámetro la varaible \texttt{page}, la cual por defecto siempre será 1.
    
    \item \textbf{Cantidad de resultados:} Se obtienen resultados en bloques de 10, por lo que por cada página de la API se obtienen 2 bloques de resultados.
\end{itemize}

Puesto que la variable \lstinline{page} que viene como parámetro no es precisamente la misma de la API, se tiene que dividir los resultados en base a si la página es par o impar, siguiendo el procedimiento:

\begin{center}
\begin{lstlisting}
      int pageIndex = (page ~/ 2);
      if(page%2 != 0) {
        pageIndex++;
      }

      // Hacer la consulta a la API...

      // Odd numbers: 1-10
      if(page%2 == 0) {
        // Retornar primer grupo
      }

      // Even numbers: 11-20
      return // Retornar segundo grupo
\end{lstlisting}
\end{center}
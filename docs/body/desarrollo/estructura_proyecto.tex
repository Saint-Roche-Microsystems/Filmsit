\subsection{Estrucutra de proyecto}

Se consume la API pública de The Movie Database (TMDb) para obtener información sobre películas, series de televisión y actores. La API proporciona datos detallados, incluyendo títulos, descripciones, calificaciones, imágenes y más.

La app utiliza el patrón arquitectónico MVVM en conjunto con Clean Architecture para organizar su distribución.Para ello, se parte de la carpeta “src/”, la cual se compone de los siguientes directorios:

\begin{figure}[H]
    \centering
    \includegraphics[width=0.7 \textwidth, height=10cm, keepaspectratio]{estructura_proyecto.png}
    \caption{Estructura de directorios}
    \label{fig:estructura_proyecto}
\end{figure}

Siendo los directorios princiales los siguientes:

\begin{itemize}
    \item \textbf{core:} Contiene la configuración principal de la aplicación, donde se optienen las variables de entorno. También se definen manejadores de errores simples, utilizados en llamadas de API u otros errores en la aplicación. 
    
    \item \textbf{data:} Implementa la capa de datos, incluyendo modelos, fuentes de datos (en este caso remotas por la API) y repositorios que gestionan la obtención y almacenamiento de datos.
    
    \item \textbf{domain:} Define la lógica de negocio a través de entidades, casos de uso a partir de los repositorios abstractos que describen las operaciones disponibles.
    
    \item \textbf{presentation:} Maneja la interfaz de usuario, el ruteo y la interacción del usuario mediante pantallas, widgets y proveedores para el estado reactivo con view models.
    
    \item \textbf{themes:} Contiene los esquemas de colores, tipografía y estilos reutilizables para mantener la consistencia visual en toda la aplicación.
\end{itemize}
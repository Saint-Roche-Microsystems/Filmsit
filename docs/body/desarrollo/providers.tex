\subsection{Declaración de Providers}

Estos se ubican dentro del archivo \lstinline{main.dart}. Se utiliza la librería “Provider” para la gestión del estado de la aplicación. Primero se parte de la definición de los providers http, datasource y repositories.

\begin{center}
\begin{lstlisting}
  final httpClient = http.Client();
  final movieDs = MovieDataSource(client: httpClient);
  final movieRep = MovieRepository(ds: movieDs);
  final genreDs = GenreDataSource(client: httpClient);
  final genreRep = GenreRepository(ds: genreDs);
\end{lstlisting}
\end{center}

Es en base a estos que luego se pueden definir los providers para cada uno de los casos de uso siguiendo la estructura:

\begin{center}
\begin{lstlisting}
Provider<CASO_DE_USO>(
    create: (_) => CASO_DE_USO(REPOSITORIO),
),
\end{lstlisting}
\end{center}

Una vez definidos los provideres de los casos de uso, se los suministra a los viewmodels correspondientes mediante los \lstinline{ChangeNotifierProvider} siguiendo la estructura:

\begin{center}
\begin{lstlisting}
ChangeNotifierProvider<VIEW_MODEL>(
  create: (context) => VIEW_MODEL(
    useCase1: context.read<useCase1>(),
    useCase2: context.read<useCase2>(),
    ...
  ),
),
\end{lstlisting}
\end{center}